% MACROS FOR CREATING bimatrix games
% registers allocated once only
\newcount\rows
\newcount\cols
\newcount\rowcoord
\newcount\colcoord
\newcount\m
\newcount\n
% the crucial variable-length-parameter macro \dosth
\def\dosth#1{\ifx###1##\else\dofirst#1\anytoken\fi}
\def\doagain#1\anytoken{\dosth{#1}}
% example of \dofirst
% \def\dofirst#1{{$\langle#1\rangle$}\doagain} 
% example of \dosth
% \dosth{1234x{x^3}y}
\def\payoffpairs#1#2#3{\m=#1\multiply\m by 4 \advance\m by -1 \n=1
  \def\dofirst##1{\put(\n,-\m){\makebox(0,0){\strut##1}}\advance\n by 4 \doagain}%
  \dosth{#2\strut}%
  \m=#1\multiply\m by 4 \advance\m by -3 \n=3 \dosth{#3\strut}}
\def\singlepayoffs#1#2{\m=#1\multiply\m by 4 \advance\m by -2 \n=2
  \def\dofirst##1{\put(\n,-\m){\makebox(0,0){\strut##1}}\advance\n by 4 \doagain}%
  {\large\dosth{#2\strut}}}
% the bimatrix game command
\newcommand{\bimatrixgame}[8]{%
\setlength{\unitlength}{#1}%
\rows=#2
\cols=#3
\rowcoord=\rows
\colcoord=\cols
\multiply\rowcoord by 4
\multiply\colcoord by 4
\m=\rowcoord
\n=\colcoord
\advance\m by 2 % 2 units left of payoff table
\advance\n by 2 % 2 units above payoff table
\begin{picture}(\n,\m)(-2,-\rowcoord)
\m=\rows
\n=\cols
\advance\m by 1
\advance\n by 1 
\thinlines
\multiput(0,0)(0,-4){\m}{\line(1,0){\colcoord}}
\multiput(0,0)(4,0){\n}{\line(0,-1){\rowcoord}}
\put(0,0){\line(-1,1){2}}
\put(-1.5,0.5){\makebox(0,0)[r]{#4}}  % name player I
\put(-.7,1.7){\makebox(0,0)[l]{#5}}   % name player II
%row annotations - even with long strategy names, stick out to the left
\n=2
\def\dofirst##1{\put(-0.8,-\n){\makebox(0,0)[r]{\strut##1}}\advance\n by 4
   \doagain}
\dosth{#6\strut} 
%column annotations
\n=2
\def\dofirst##1{\put(\n,1.0){\makebox(0,0){\strut##1}}\advance\n by 4
   \doagain}
\dosth{#7\strut}#8%
\end{picture}}
%
% example usage:
% \def\mm#1{\makebox(0,0){\strut#1}}% 
%
% \bimatrixgame{4mm}{3}{4}{I}{II}{TMB}{lcr{\it out}}
% {
% \payoffpairs{1}{00{$a^2$}0}{1{\fbox 3}{\fbox 3}2}
% \payoffpairs{2}{0000}{1111}
% \singlepayoffs{3}{5555}
% % \multiput(0,-0.13)(.16,-.16){75}{\tiny.} 
% \put(10,-2){\mm{*}}
% } 
